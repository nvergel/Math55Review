\documentclass[12pt]{article}

\usepackage{fullpage}
\pagestyle{empty}

\usepackage{amsmath, amsfonts, amssymb}
\usepackage{mathtools}

\def\multiset#1#2{\ensuremath{\left(\kern-.3em\left(\genfrac{}{}{0pt}{}{#1}{#2}\right)\kern-.3em\right)}}
\def \z{\mathbb{Z}}
\def \n{\mathbb{N}}
\def \no{\noindent}
\def \p{\pagebreak}
\def \et{\phi (n)}
\def \m{\equiv_m}

\begin{document}

\begin{center} Recurrence Relations \end{center}

\no Given a kth order recurrence of the form $a_n=x_1a_{n-1}+x_2a_{n-2}$, what is:\\
\begin{itemize}
	\item[i] The characteristic polynomial:
	\item[ii] The closed form solution:
\end{itemize}

\begin{center} Theorems \end{center}

\no  ``I.C. Theorem":\\

\no  Bezout's Theorem:\\

\no Corollary to Bezout's Theorem:\\

\no The Euclidean Algorithm:\\

\no  ``Important" Theorem:\\

\no  Prime Importance:\\

\no  Fundamental Theorem of Arithmetic:\\
\begin{itemize}
	\item[i] $ (a, b) = $
	\item[ii] $ [a,b] = $
\end{itemize}


\begin{center} Modular Arithmetic \end{center}


\no Congruence:\\

\no Cancellation Theorem:\\

\no Wilson's Theorem:\\

\no Fermat's Theorem:\\

\no Euler's Totient Function:\\
\begin{itemize}
	\item[i] What does it tell use?
	\item[ii] How do we calculate it?\\
\end{itemize}

\no Euler's Theorem:\\

\end{document}
