\documentclass[12pt]{article}

\usepackage{fullpage}
\pagestyle{empty}

\usepackage{amsmath, amsfonts, amssymb}
\usepackage{mathtools}

\def\multiset#1#2{\ensuremath{\left(\kern-.3em\left(\genfrac{}{}{0pt}{}{#1}{#2}\right)\kern-.3em\right)}}
\def \z{\mathbb{Z}}
\def \n{\mathbb{N}}
\def \no{\noindent}
\def \p{\pagebreak}

\begin{document}

\begin{center} Definitions \end{center}

\no sets are denoted with \{ ... \}\\

\no $\z$ set of all integers \{..., -2, -1, 0, 1, 2, ...\}\\

\no $\n$ set of all natural numbers \{0, 1, 2, ...\}\\

\no \textbackslash \{x\} means x is not in the set ie. \{1, 2, ...\} can be denoted as $\n$\textbackslash$\{0\}$\\

\no $\in$ - element in set\\

\no $\exists$ - there exists\\

\no $\forall$ - for all\\

\no If $p$ is not prime then it is composite. 0 and 1 are neither prime nor composite\\

\no 

\no

\begin{center} Recurrence Relations \end{center}

Given a kth order recurrence of the form $a_n=x_1a_{n-1}+x_2a_{n-2}+...+x_ka_{n-k}$, the closed form is $a_n = C_1r_1^n+C_2r_2^n+...+C_kr_k^n$ where $r_1,r_2,...,r_k$ are solutions to $r^k-x_1r^{k-1}-x_2r^{k-2}-...-x_k=0$.\\

For example fibonacci numbers, $F_n = F_{n-1} + F_{n-2}$, have the characteristic polynomial equation, $r^2 -r -1 = 0$, and the solutions to the equation are $r=\frac{1+\sqrt{5}}2$ and $r=\frac{1-\sqrt{5}}2$.

\begin{center} Theorems \end{center}

\no Addition, multiplication and subtraction are closed $\forall x;$ $x \in \z$.\\

\no THM: The Division Algorithm\\
\hangindent=1cm For $\frac ab$ where $a \in \z$ and $b \in \n$\textbackslash$\{0\}$, there is some quotient $q \in \n$ and remainder $0 \leq r < b$ such that, \[ a = bq + r.\]
 \begin{itemize}
	\item[i] $b$ is said to divide a if $r=0$. We denote this as $b|a$. If $b$ does not divide $a$ we say $b \nmid a$.
	\item[ii] When (a, b)=1, a and b are relatively prime\\
\end{itemize}

\no THM: Integer Combination ``I.C Theorem"\\
\hangindent=1cm If $d|a$ and $d|b$ then $d|(ax + by)$ for $x,y \in \z$.\\

\no Greatest Common Divisor:\\
\hangindent=1cm We denote gcd(x,y) simply as (x,y). Note that (x,0) = x.\\

\no THM: Bezout's Theorem:\\
\hangindent=1cm For $a,b \in \z$, (a,b) is the smallest positive integer of the form $ax+by$ where $x,y \in \z$.\\

\no THM: Euclid's Theorem:\\
\hangindent=1cm For any integers, $a, b, x: (a, b) = (b, a - bx)$\\

\no The Euclidean Algorithm:\\
\hangindent=1cm (a, b)=(b, a mod b) where $b<a$\\

\no Corollary to Bezout's Lmema:\\
\hangindent=1cm If (a,b)=1 then there exist some $x,y \in \z$ such that $ax + by = 1$\\

\no THM: "Important" Theorem:\\
\hangindent=1cm  If d|ab and (d,a) = 1 then d|b\\

\no THM: Fundamental Theorem of Arithmetic:\\
\hangindent=1cm Every integer n has a unique factorization into primes. For example $1200 = 2^4 3^1 5^2$\\



\end{document}
