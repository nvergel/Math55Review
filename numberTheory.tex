\documentclass[12pt]{article}

\usepackage{fullpage}
\pagestyle{empty}

\usepackage{amsmath, amsfonts, amssymb}
\usepackage{mathtools}

\def\multiset#1#2{\ensuremath{\left(\kern-.3em\left(\genfrac{}{}{0pt}{}{#1}{#2}\right)\kern-.3em\right)}}
\def \z{\mathbb{Z}}
\def \r{\mathbb{R}}
\def \n{\noindent}
\def \p{\pagebreak}

\begin{document}

\begin{center} Notes on notation \end{center}

\n sets are denoted with \{ ... \}\\

\n $\z$ set of all integers \{..., -2, -1, 0, 1, 2, ...\}\\

\n $\r$ set of all natural numbers \{0, 1, 2, ...\}\\

\n \textbackslash \{x\} means x is not in the set ie. \{1, 2, ...\} can be denoted as $\r$\textbackslash$\{0\}$\\

\n $\in$ - element in set\\

\n $\exists$ - there exists\\

\n $\forall$ - for all

\begin{center} Theorems \end{center}

\n Addition, multiplication and subtraction are closed $\forall x;$ $x \in \z$.\\

\n THM: The Division Algorithm\\
\hangindent=1cm For $\frac ab$ where $a \in \z$ and $b \in \r$\textbackslash$\{0\}$, there is some quotient $q \in \r$ and remainder $0 \leq r < b$ such that,
\[ a = bq + r\]
$b$ is said to divide a if $r=0$. We denote this as $b|a$. If $b$ does not divide $a$ we say $b \nmid a$.\\

\n THM: Integer Combination ``I.C Theorem"\\
\hangindent=1cm If $d|a$ and $d|b$ then $d|(ax + by)$ for $x,y \in \z$.\\

\n Greatest Common Divisor: We denote gcd(x,y) simply as (x,y)\\
\hangindent=1cm Notes to self: if in doubt, (x,y) = 1 (ie. 1 is best guess when you are too lazy too compute gcd). Also, (x,0) = x.\\

\n THM: Benzout's theorem:\\
\hangindent=1cm For $a,b \in \z$, (a,b) is the smallest positive integer of the form $ax+by$ where $x,y \in \z$.\\



\end{document}
