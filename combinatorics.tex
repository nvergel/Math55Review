\documentclass[12pt]{article}

\usepackage{fullpage}
\pagestyle{empty}

\usepackage{amsmath}
\usepackage{mathtools}
\def\multiset#1#2{\ensuremath{\left(\kern-.3em\left(\genfrac{}{}{0pt}{}{#1}{#2}\right)\kern-.3em\right)}}

\begin{document}

\begin{center} n chose k \end{center}

\noindent Number of ways to chose k elements from a set of n elements when order does not matter and\\

\begin{itemize}
\item[i] there is no replacement:

\[ \binom{n}{k} = \frac{n!}{k!(n-k)!} \]

\item[ii] there is replacement (multichose):
\[\multiset{n}{k} = \binom{n+k-1}{k}\]\\
\end{itemize}

\noindent Pascal's identity:
\[\binom{n}{k} = \binom{n-1}{k}  + \binom{n-1}{k-1} \]

\noindent Similarly:
\[\multiset{n}{k} = \multiset{n-1}{k} + \multiset{n}{k-1}\]\\

\noindent Hockey stick identity:
\[\sum_{x=k}^n{\binom{x}{k}} = \binom{n+1}{k+1}\]\\

\noindent Binomial identity:
\[(x+y)^n = \sum{\binom{n}{k} x^k y^{n-k}} \]\\

\noindent Other properties:
\[\binom{n}{k} = \binom{n}{n-k} \]
\[\sum{\binom{n}{k}} = 2^n\]

\pagebreak

\begin{center} Fibonacci Numbers \end{center}

\noindent Recursive definition:

\[F_0 = 0\]
\[F_1 = 1\]
\[F_n = F_{n-1} + F_{n-1} \]

\noindent Binet's formula:

\[F_n = \frac 1{\sqrt{5}} (\phi^n - \bar{\phi}^n)\]
\[\phi = \frac{1+\sqrt{5}}2 \text{ and } \bar{\phi} = \frac{1-\sqrt{5}}2\]

\begin{center}  Note that $\phi$ and $\bar{\phi}$ are solutions to $x^2 = x + 1$ \end{center}

\noindent First 13 numbers of Fibonacci series:
\[0, 1, 1, 2, 3, 5, 8, 13, 21, 34, 55, 89, 144\]\\

\noindent All the ways ($f_n$) to tile a board of length $1 \times n$ with squares ($1 \times 1$) and dominoes ($1 \times 2$):
\[ f_n = F_{n+1}\]

\pagebreak

\begin{center} Sets \end{center}

\noindent The union of two sets, $A \cup B$, is the set of all elements contained in both A or B.\\

\noindent The intersection of two sets, $A \cap B$ or simply $AB$, is the set of all elements in A and B.\\

\noindent Two sets are disjoint if $A \cap B $ is the null set $\{\}$\\

\noindent The size of a set $A$ is denoted by $|A|$\\

\noindent Rule of sum: given disjoint sets $A_1, A_2, A_3, ..., A_n$, the size of there union is
\[|A_1 \cup A_2 \cup A_3 \cup ... \cup A_n| = \sum{|A_i|} -  \sum \sum{|A_iA_j|} + \sum \sum \sum{|A_iA_jA_k|} - ... - (-1)^n |A_iA_jA_k...A_n|\]







\end{document}
